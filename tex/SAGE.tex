%%%%%%%%%%%%%%%%%%%%%%%%%%%%%%%%%%%%%%%% Klasse Festlegen
\documentclass[Master,BMR,english]{BASE/twbook} 
%\documentclass[Bachelor,BMR,english,fhCitStyle,IEEE]{BASE/twbook} % FH definierte Zitierstandards verwenden 
%%%%%%%%%%%%%%%%%%%%%%%%%%%%%%%%%%%%%%%% Verwendete Packages
\usepackage[utf8]{inputenc} % Zeichen-Enkodierung (evtl. Abweichungen für Apple)
\usepackage[T1]{fontenc}    % Zeichen-Enkodierung
\usepackage{blindtext}      % Platzhaltertexte
\usepackage{minted}         % Darstellung von Code
\usepackage{comment}        % Auskommentieren von ganzen Passagen
\usepackage{csquotes}
\usepackage{algorithm}      % Umgebung f Algorithmen
\usepackage[noend]{algpseudocode}
                            % Wenn Sie während Ihrer Arbeit
                            % merken, dass Sie zusätzliche Funktionen
                            % benötigen ist hier ein guter Platz um
                            % weitere Packages zu laden
%%%%%%%%%%%%%%%%%%%%%%%%%%%%%%%%%%%%%%%% Zitierstil zum selbst definieren
\usepackage[backend=biber, style=ieee]{biblatex}            % LaTeX definierter IEEE- Standard
%\usepackage[backend=biber, style=authoryear]{biblatex}      % LaTeX definierter Harvard-Standard
\addbibresource{Literatur.bib}                              % Literatur-File definieren
%%%%%%%%%%%%%%%%%%%%%%%%%%%%%%%%%%%%%%%% Einträge für Deckblatt
\title{Arbeitstitel\\Arbeitstitel}

\author{Kevin Eppacher, Titel}
\studentnumber{XXXXXXXXXXXXXXX}
%\author{Titel Vorname Name, Titel\and{}Titel Vorname Name, Titel}
%\studentnumber{XXXXXXXXXXXXXXX\and{}XXXXXXXXXXXXXXX}

\supervisor{Titel Vorname Name, Titel}
%\supervisor[Begutachter]{Titel Vorname Name, Titel}
%\supervisor[Begutachterin]{Titel Vorname Name, Titel}
%\secondsupervisor{Titel Vorname Name, Titel}
%\secondsupervisor[Begutachter]{Titel Vorname Name, Titel}
%\secondsupervisor[Begutachterinnen]{Titel Vorname Name, Titel}

\place{Wien}
%%%%%%%%%%%%%%%%%%%%%%%%%%%%%%%%%%%%%%%% Danksagung/Kurzfassung/Schlagworte
\kurzfassung{\blindtext}
\schlagworte{Schlagwort1, Schlagwort2, Schlagwort3, Schlagwort4}
\outline{\blindtext}
\keywords{Keyword1, Keyword2, Keyword3, Keyword4}
\acknowledgements{\blindtext}
\setListingsAndAcronyms % Definition der Namen für Quellcodeverzeichnis 
%%%%%%%%%%%%%%%%%%%%%%%%%%%%%%%%%%%%%%%% Ende des Headers
%%%%%%%%%%%%%%%%%%%%%%%%%%%%%%%%%%%%%%%% Beginn des Dokuments
\begin{document}


\maketitle
%%%%%%%%%%%%%%%%%%%%%%%%%%%%%%%%%%%%%%%% Beginn des Inhalts




%%%%%%%%%%%%%%%%%%%%%%%%%%%%%%%%%%%%%%%%%%%%%%%%%%%%%%%%%%%%%%%%%% Hier beginnen die Verzeichnisse.
\clearpage                                                       % Beginne neue Seite

\printbib                                                        % Literaturverzeichnis LaTeX-Zitier-Standard
%\printbib{Literatur}                                             % Literaturverzeichnis FH-Zitier-Standard
\clearpage

\listoffigures                                                   % Abbildungsverzeichnis
\clearpage

\listoftables                                                    % Tabellenverzeichnis
\clearpage

\listoflistings                                                  % Quellcodeverzeichnis
\clearpage

\phantomsection
\addcontentsline{toc}{chapter}{\listacroname}
\chapter*{\listacroname}
\begin{acronym}[XXXXX]
    \acro{ABC}[ABC]{Alphabet}
    \acro{WWW}[WWW]{world wide web}
    \acro{ROFL}[ROFL]{Rolling on floor laughing}
\end{acronym}
%%%%%%%%%%%%%%%%%%%%%%%%%%%%%%%%%%%%%%%%%%%%%%%%%%%%%%%%%%%%%%%%%% Hier beginnt der Anhang.
\clearpage
\appendix
\chapter{Anhang A}
\clearpage
\chapter{Anhang B}
\end{document}
%%%%%%%%%%%%%%%%%%%%%%%%%%%%%%%%%%%%%%%%%%%%%%%%%%%%%%%%%%%%%%%%%% Ende des Inhalts
