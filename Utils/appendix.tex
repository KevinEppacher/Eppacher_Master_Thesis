%%%%%%%%%%%%%%%%%%%%%%%%%%%%%%%%%%%%%%%% Appendix
\clearpage
\appendix
\chapter{Methodology details}
\label{ch:appendix:methodology_details}
\section{Behavior Tree Structure}

\begin{figure}[h!]
    \centering
    \includegraphics[width=0.8\textwidth]{Images/03_methods/bt/sage_behaviortree.png}
    \caption{Behavior Tree structure of the proposed hybrid semantic exploration system. The BT alternates between detection and exploration branches based on object detection status, guiding the robot's navigation and observation strategies.}
    \label{fig:appendix:bt:tree}
\end{figure}

\chapter{Additional Experimental Results}
\label{ch:appendix:experimental_results}

% ---------------------------------------------------

\begin{figure}[h!]
    \centering
    \begin{minipage}[t]{0.48\textwidth}
        \centering
        \includegraphics[width=\textwidth]{Images/05_experimental_results/RQ1/00848-ziup5kvtCCR/bed.png}
        \subcaption{Object detection overlay (bed)}
        \label{fig:appendix:rq1:00848:bed}
    \end{minipage}
    \hfill
    \begin{minipage}[t]{0.48\textwidth}
        \centering
        \includegraphics[width=\textwidth]{Images/05_experimental_results/RQ1/00848-ziup5kvtCCR/bed_paths.png}
        \subcaption{Navigation trajectory}
        \label{fig:appendix:rq1:00848:bed_paths}
    \end{minipage}

    \caption{RQ1 example (scene ziup5kvtCCR): object detection overlay and executed navigation trajectory for the target object \emph{bed}.}
    \label{fig:appendix:rq1:00848:bed_combined}
\end{figure}

% ---------------------------------------------------

\begin{figure}[h!]
    \centering
    \begin{minipage}[t]{0.48\textwidth}
        \centering
        \includegraphics[width=\textwidth]{Images/05_experimental_results/RQ1/00800-TEEsavR23oF/bed.png}
        \subcaption{Object detection overlay (bed)}
        \label{fig:appendix:rq1:00800:bed}
    \end{minipage}
    \hfill
    \begin{minipage}[t]{0.48\textwidth}
        \centering
        \includegraphics[width=\textwidth]{Images/05_experimental_results/RQ1/00800-TEEsavR23oF/bed_paths.png}
        \subcaption{Navigation trajectory}
        \label{fig:appendix:rq1:00800:bed_paths}
    \end{minipage}

    \caption{RQ1 example (scene TEEsavR23oF): object detection overlay and executed navigation trajectory for the target object \emph{bed}.}
    \label{fig:appendix:rq1:00800:bed_combined}
\end{figure}

% ---------------------------------------------------

\begin{figure}[h!]
    \centering
    \begin{minipage}[t]{0.48\textwidth}
        \centering
        \includegraphics[width=\textwidth]{Images/05_experimental_results/RQ1/00814-p53SfW6mjZe/toilet.png}
        \subcaption{Object detection overlay (toilet)}
        \label{fig:appendix:rq1:00814:toilet}
    \end{minipage}
    \hfill
    \begin{minipage}[t]{0.48\textwidth}
        \centering
        \includegraphics[width=\textwidth]{Images/05_experimental_results/RQ1/00814-p53SfW6mjZe/toilet_paths.png}
        \subcaption{Navigation trajectory}
        \label{fig:appendix:rq1:00814:toilet_paths}
    \end{minipage}

    \caption{RQ1 example (scene p53SfW6mjZe): object detection overlay and executed navigation trajectory for the target object \emph{toilet}.}
    \label{fig:appendix:rq1:00814:toilet_combined}
\end{figure}

% ---------------------------------------------------

\begin{figure}[h!]
    \centering
    \begin{minipage}[t]{0.48\textwidth}
        \centering
        \includegraphics[width=\textwidth]{Images/05_experimental_results/RQ1/00824-Dd4bFSTQ8gi/toilet.png}
        \subcaption{Object detection overlay (toilet)}
        \label{fig:appendix:rq1:00824:toilet}
    \end{minipage}
    \hfill
    \begin{minipage}[t]{0.48\textwidth}
        \centering
        \includegraphics[width=\textwidth]{Images/05_experimental_results/RQ1/00824-Dd4bFSTQ8gi/toilet_paths.png}
        \subcaption{Navigation trajectory}
        \label{fig:appendix:rq1:00824:toilet_paths}
    \end{minipage}

    \caption{RQ1 example (scene Dd4bFSTQ8gi): object detection overlay and executed navigation trajectory for the target object \emph{toilet}.}
    \label{fig:appendix:rq1:00824:toilet_combined}
\end{figure}

% ---------------------------------------------------

\begin{figure}[h!]
    \centering
    \begin{minipage}[t]{0.48\textwidth}
        \centering
        \includegraphics[width=\textwidth]{Images/05_experimental_results/RQ1/00824-Dd4bFSTQ8gi/plant.png}
        \subcaption{Object detection overlay (plant)}
        \label{fig:appendix:rq1:00824:plant}
    \end{minipage}
    \hfill
    \begin{minipage}[t]{0.48\textwidth}
        \centering
        \includegraphics[width=\textwidth]{Images/05_experimental_results/RQ1/00824-Dd4bFSTQ8gi/plant_paths.png}
        \subcaption{Navigation trajectory}
        \label{fig:appendix:rq1:00824:plant_paths}
    \end{minipage}

    \caption{RQ1 example (scene Dd4bFSTQ8gi): object detection overlay and executed navigation trajectory for the target object \emph{plant}.}
    \label{fig:appendix:rq1:00824:plant_combined}
\end{figure}

% ---------------------------------------------------

\begin{figure}[h!]
    \centering
    \begin{minipage}[t]{0.48\textwidth}
        \centering
        \includegraphics[width=\textwidth]{Images/05_experimental_results/RQ1/00848-ziup5kvtCCR/tree.png}
        \subcaption{Object detection overlay (tree)}
        \label{fig:appendix:rq1:00848:tree}
    \end{minipage}
    \hfill
    \begin{minipage}[t]{0.48\textwidth}
        \centering
        \includegraphics[width=\textwidth]{Images/05_experimental_results/RQ1/00848-ziup5kvtCCR/tree_paths.png}
        \subcaption{Navigation trajectory}
        \label{fig:appendix:rq1:00848:tree_paths}
    \end{minipage}

    \caption{RQ1 example (scene ziup5kvtCCR): object detection overlay and executed navigation trajectory for the target object \emph{tree}.}
    \label{fig:appendix:rq1:00848:tree_combined}
\end{figure}

% ---------------------------------------------------

\begin{figure}[h!]
    \centering
    \begin{minipage}[t]{0.48\textwidth}
        \centering
        \includegraphics[width=\textwidth]{Images/05_experimental_results/RQ1/00876-mv2HUxq3B53/sofa.png}
        \subcaption{Object detection overlay (sofa)}
        \label{fig:appendix:rq1:00876:sofa}
    \end{minipage}
    \hfill
    \begin{minipage}[t]{0.48\textwidth}
        \centering
        \includegraphics[width=\textwidth]{Images/05_experimental_results/RQ1/00876-mv2HUxq3B53/sofa_paths.png}
        \subcaption{Navigation trajectory}
        \label{fig:appendix:rq1:00876:sofa_paths}
    \end{minipage}

    \caption{RQ1 example (scene mv2HUxq3B53): object detection overlay and executed navigation trajectory for the target object \emph{sofa}.}
    \label{fig:appendix:rq1:00876:sofa_combined}
\end{figure}

% ---------------------------------------------------

\begin{figure}[h!]
    \centering
    \begin{minipage}[t]{0.48\textwidth}
        \centering
        \includegraphics[width=\textwidth]{Images/05_experimental_results/RQ1/00876-mv2HUxq3B53/chair.png}
        \subcaption{Object detection overlay (chair)}
        \label{fig:appendix:rq1:00876:chair}
    \end{minipage}
    \hfill
    \begin{minipage}[t]{0.48\textwidth}
        \centering
        \includegraphics[width=\textwidth]{Images/05_experimental_results/RQ1/00876-mv2HUxq3B53/chair_paths.png}
        \subcaption{Navigation trajectory}
        \label{fig:appendix:rq1:00876:chair_paths}
    \end{minipage}

    \caption{RQ1 failure example (scene mv2HUxq3B53): false positive detection caused by a visually correct but physically unreachable object. The chair is correctly detected in the image but is located behind a window outside the navigable space, leading the robot to navigate towards the window under the assumption that the object is reachable.}
    \label{fig:appendix:rq1:00876:chair_combined}
\end{figure}

% ---------------------------------------------------

\begin{figure}[h!]
    \centering
    \includegraphics[width=\textwidth]{Images/05_experimental_results/RQ5/openfusion_max_voxel.png}
    \caption{Example semantic map built with a total voxel count of 191,941 using OpenFusion with a maximum voxel limit of 200,000. The map effectively captures the environment's structure and semantics within the specified memory constraints.}
    \label{fig:appendix:rq5:openfusion_max_voxel}
\end{figure}

% ---------------------------------------------------

\begin{figure}[h!]
    \centering
    \includegraphics[width=\textwidth]{Images/06_appendix/semantic_evaluation_pointcloud.png}
    \caption{Semantic evaluation point cloud generated with OpenFusion~\cite{yamazakiOpenFusionRealtimeOpenVocabulary2023}. The point cloud is constructed using class predictions provided by Matterport3D~\cite{ramakrishnanHabitatMatterport3DDataset2021} and subsequently filtered by object classes, clustered, and reduced to object centroids. These centroids serve as ground truth object locations for quantitative evaluation.}
    \label{fig:appendix:semantic_evaluation_pointcloud}
\end{figure}