\chapter{Experimental Results}
This chapter presents the evaluation results of the proposed hybrid semantic exploration framework. The experiments are designed to address the defined research contributions by quantitatively and qualitatively assessing system performance in both simulated and real-world scenarios.

\section{Evaluation Metrics}
\begin{itemize}
    \item \textbf{Success Rate (SR):} Ratio of successfully reached goal objects during semantic search.
    \item \textbf{Multi-Object Success Rate:} Percentage of completed search tasks involving multiple target objects.
    \item \textbf{Success weighted by Path Length (SPL):} Efficiency of navigation paths considering both success and path optimality.
    \item \textbf{Semantic Precision:} Accuracy of object recognition and detection under open-vocabulary queries.
    \item \textbf{Robustness:} Qualitative metric reflecting the system's tolerance to false positives, partial occlusions, and noisy sensor data.
    \item \textbf{System Performance:} Real-time metrics such as CPU/GPU load, average FPS, and processing latency per cycle.
\end{itemize}

\section{SC1: Multi-Object Search Efficiency and Success Rate}
\begin{itemize}
    \item Evaluation of single- and multi-object search scenarios using benchmark datasets.
    \item Comparison of success rate (SR) and multi-object SR with existing methods such as OneMap, VLMaps, and GeFF.
    \item Assessment of path efficiency using SPL metric across identical navigation tasks.
\end{itemize}

\section{SC2: Resource Efficiency through Lightweight Architecture}
\begin{itemize}
    \item Evaluation of memory usage and model inference time using SEEM in place of multi-model VLFM pipeline.
    \item Comparison of GPU memory footprint and runtime on equivalent hardware setups.
    \item Qualitative discussion on model simplification trade-offs versus performance.
\end{itemize}

\section{SC3: Fusion Strategy and Robustness to False Positives}
\begin{itemize}
    \item Testing the sensor-based fusion approach under noisy and ambiguous object detections.
    \item Analysis of semantic precision before and after fusion of VLFM and Open-Fusion detections.
    \item Ablation study comparing performance with and without fusion, especially under occlusion and false positive conditions.
\end{itemize}

\section{SC4: Real-World System Evaluation}
\begin{itemize}
    \item Deployment on a mobile robot in real indoor environments with varying lighting and sensor noise.
    \item Measurement of real-time capabilities: FPS, inference latency, and CPU/GPU load.
    \item Qualitative assessment of robustness in dynamic, unstructured scenes.
\end{itemize}
