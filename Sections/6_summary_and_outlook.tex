% chktex-file 44

\chapter{Summary and Outlook}
\label{ch:conclusion}

Open-vocabulary semantic exploration in unstructured environments represents a fundamental challenge at the intersection of robotics, computer vision, and natural language understanding. Autonomous agents must not only navigate efficiently through unknown spaces, but also reason about semantically meaningful targets specified at runtime, under conditions of partial observability, perceptual ambiguity, and limited computational resources.

Prior work in semantic exploration has primarily followed two directions. Reinforcement learning-based approaches achieve strong performance through extensive training in simulation~\cite{majumdarZSONZeroShotObjectGoal2023,ramakrishnanPONIPotentialFunctions2022,ramrakhyaPIRLNavPretrainingImitation2023}, but often suffer from limited generalization and high training costs. In contrast, recent zero-shot methods leverage \acp{VLM} to guide exploration toward semantically relevant regions without task-specific training~\cite{yokoyamaVLFMVisionLanguageFrontier2023,gadreCoWsPastureBaselines2022}. To further improve efficiency, several frameworks incorporate persistent semantic maps to accumulate knowledge over time and reduce redundant observations~\cite{buschOneMapFind2025,huangVisualLanguageMaps2023}. However, existing methods either rely exclusively on semantic maps for decision-making or omit persistent mapping entirely, which can lead to brittle behavior, inefficient exploration, or limited robustness to perceptual noise. Moreover, many approaches do not explicitly fuse multi-source semantic evidence, leaving object detection performance vulnerable to false negatives and inconsistent observations.

This thesis introduced \textbf{SAGE}, a hybrid open-vocabulary semantic exploration framework that combines frontier-based exploration with persistent semantic memory and vision-language reasoning. By integrating reactive semantic frontier scoring with accumulated semantic maps, \textbf{SAGE} balances exploration of unseen space and exploitation of previously acquired knowledge. A multi-source fusion strategy further enhances robustness by combining detection confidence, vision-language similarity, and temporal memory evidence, allowing the system to reinforce consistent semantic hypotheses across viewpoints and time.

Experiments were conducted in IsaacSim on HM3Dv2 scenes~\cite{ramakrishnanHabitatMatterport3DDataset2021} using an evaluation pipeline reporting common object-goal navigation metrics (e.g., \ac{SR} and \ac{SPL}) while operating outside the HabitatSim stack. \textbf{SAGE} achieves strong performance across scenes and start states, and the ablation studies: (i) balancing frontier-driven exploration and memory exploitation improves stability in the presence of noisy semantic cues, (ii) costmap-informed frontier scoring reduces navigation failures in cluttered regions (see Chapter~\ref{ch:results} Section~\ref{sec:experiments:results:3}), and (iii) multi-source semantic fusion substantially reduces false negatives in object confirmation, improving downstream decision-making despite a moderate increase in false positives.

Beyond performance, this work emphasizes system efficiency and practical deployability. Compared to vision-language frontier pipelines that instantiate multiple large perception models (e.g., VLFM~\cite{yokoyamaVLFMVisionLanguageFrontier2023}), \textbf{SAGE} employs a streamlined set of GPU-intensive components. The exploration-related perception modules require approximately 3.5,GB of GPU memory in the presented configuration, and with a medium-sized semantic map of roughly 200{,}000 voxels the total footprint remains below 10,GB. This resource profile supports deployment on consumer-grade GPUs and enables distributed configurations in which the semantic memory can be offloaded to a separate compute unit.

Despite these results, several avenues for future work remain. First, the semantic mapping backbone could be replaced or augmented with more advanced open-vocabulary mapping frameworks that provide improved fusion quality and uncertainty handling, such as OTAS~\cite{schwaigerOTASOpenvocabularyToken2025} or DualMap~\cite{jiangDualMapOnlineOpenVocabulary2025}. Second, the exploration-exploitation balance in \textbf{SAGE} is currently governed by a fixed hyperparameter. Adaptive strategies that adjust this balance online based on environmental complexity, task progress, or perceptual confidence could further improve robustness and efficiency. Finally, higher-level task reasoning could be incorporated by leveraging \acp{LLM} through existing ROS~2 action interfaces. Such models could generate sub-goals, adjust decision thresholds, or modulate exploration behavior dynamically, enabling more complex and long-horizon tasks to be executed autonomously.

\textbf{\ac{SAGE}} combines frontier-based semantic exploration with persistent 3D semantic memory and multi-source confidence fusion. In the presented IsaacSim~\cite{zhouBuildingAICPSNVIDIA2024} evaluation on HM3Dv2 scenes~\cite{ramakrishnanHabitatMatterport3DDataset2021}, performance is reported using the standard object-goal navigation metrics \ac{SR} and \ac{SPL}~\cite{puigHabitat30CoHabitat2023}. The ablations indicate that (a) intermediate exploration-memory weighting reduces sensitivity to semantic noise in the semantic map, (b) costmap-informed frontier scoring reduces navigation-related failures in cluttered regions, and (c) multi-source fusion improves the precision-recall trade-off of object confirmation under conservative thresholds. In addition, GPU memory usage and measured loop frequencies are reported to characterize the computational footprint.
