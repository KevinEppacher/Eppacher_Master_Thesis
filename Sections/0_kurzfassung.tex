Offen-vokabularige semantische Erkundung erfordert, dass mobile Roboter nutzerspezifische Zielobjekte in zuvor unbekannten Umgebungen finden, während sie mit partieller Beobachtbarkeit, perzeptueller Mehrdeutigkeit und begrenzten Onboard-Rechenressourcen umgehen. Bestehende Ansätze basieren entweder auf Reinforcement-Learning mit umfangreichem Simulationstraining oder nutzen Zero-Shot Vision-Language-Modelle ohne persistentes Gedächtnis, was zu ineffizienter Erkundung und fragilen Entscheidungen bei verrauschten Beobachtungen führen kann. Diese Arbeit präsentiert \textbf{SAGE} (Semantic-Aware Guided Exploration), ein hybrides Verfahren für semantische Erkundung, das (i) Frontier-basierte semantische Erkundung auf Basis von Vision-Language-Ähnlichkeit, (ii) ein persistentes dreidimensionales semantisches Gedächtnis auf Grundlage von OpenFusion zur langfristigen Exploitation sowie (iii) eine Fusion mehrerer semantischer Quellen zur Zielbestätigung kombiniert. Die Methode integriert geometrische Machbarkeit und Navigationsrestriktionen in die Frontier-Bewertung und fusioniert Detektor-Konfidenz, Vision-Language-Ähnlichkeit und Evidenz aus dem semantischen Gedächtnis mittels einer Noisy-Or-Formulierung, um insbesondere bei konservativen Entscheidungsschwellen verpasste Detektionen zu reduzieren.

\textbf{SAGE} wird in einem photorealistischen IsaacSim-Setup auf Szenen des Datensatzes Habitat-Matterport3D unter Verwendung eines Mehrziel-Suchprotokolls evaluiert. Die Experimente quantifizieren (1) die End-to-End-Navigationsleistung gegenüber ausgewählten Baselines, (2) die Sensitivität gegenüber der Gewichtung zwischen Erkundung und Gedächtnisnutzung, (3) die Robustheit gegenüber der Granularität der semantischen Karte über die OpenFusion Top-$k$-Retrievaltiefe sowie (4) den Effekt der Fusion mehrerer semantischer Quellen auf die Präzisions-Recall-Eigenschaften der Detektion. Die Ergebnisse zeigen im definierten Evaluationssetting, dass \textbf{SAGE} die höchste pfadlängengewichtete Erfolgsrate unter den verglichenen Methoden erreicht, bei gleichzeitig konkurrenzfähiger Erfolgsrate. Weitere Analysen zeigen, dass mittlere Gewichtungen zwischen Erkundung und Gedächtnisnutzung den stabilsten Kompromiss zwischen Reaktivität und Exploitation liefern und dass balancierte Erkundung die Auswirkungen verrauschter oder übermäßig dichter semantischer Karten abmildert. Die Fusion mehrerer semantischer Quellen reduziert falsch-negative Detektionen an konservativen Betriebspunkten deutlich, bei gleichzeitig erhaltener Präzision, und unterstützt damit eine zweistufige Verifikationsstrategie im Behavior Tree. Abschließend berichtet eine deploymentsorientierte Analyse den Grafikprozessor-Speicherverbrauch sowie Laufzeitkennzahlen zentraler Pipeline-Komponenten.