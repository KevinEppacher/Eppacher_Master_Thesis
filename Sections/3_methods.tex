% chktex-file 44

\chapter{Methods}
This chapter details the methods developed for semantic exploration, persistent 3D mapping, promptable object detection, and robust fusion strategies for multi-object search. 

\section{System Overview}
\begin{itemize}
    \item Presentation of the overall architecture of the exploration~(\ref{sec:semantic_frontier}), detection~(\ref{sec:zero_shot_detection}), mapping~(\ref{sec:persistent_mapping}), and fusion~(\ref{sec:fusion_strategy}) pipeline.
    \item Description of data flow between exploration (frontier evaluation), detection (promptable models), and exploitation (persistent semantic mapping), as shown in Figure~\ref{fig:system_overview}.
    \item Explanation of how exploration and mapping components interact to progressively build a semantic understanding of the environment.
\end{itemize}

\begin{figure}[h!]
    \centering
    \includegraphics[width=\textwidth]{Images/03_methods/sage_pipeline.png}
    \caption{System architecture}
    \label{fig:system_overview}
\end{figure}

\section{Semantic Frontier Exploration} \label{sec:semantic_frontier}
\textbf{Exploration 2D Occupancy Map}

\begin{itemize}
    \item The SLAM map is used for navigation.
    \item Generating frontiers for each prompt would require deleting and rebuilding the SLAM map.
    \item This approach is inefficient and impractical for navigation.
    \item Therefore, a separate 2D occupancy grid is created exclusively for exploration.
    \item The exploration map is constructed by:
    \begin{itemize}
        \item collecting robot poses, and
        \item performing raytracing against the parent SLAM map.
    \end{itemize}
    \item For each new semantic prompt:
    \begin{itemize}
        \item all stored poses are cleared, and
        \item the exploration occupancy grid is rebuilt from scratch.
    \end{itemize}
\end{itemize}

\textbf{Frontier Detection and Calculation}
\begin{itemize}
    \item Detection of frontiers on a 2D occupancy grid to identify candidate regions for exploration.
    \item Application of classical frontier-based exploration algorithms extended with semantic information.
\end{itemize}

\textbf{Value Map Generation using Vision-Language Models}
\begin{itemize}
    \item Computation of value maps by evaluating cosine similarity between text queries and scene observations.
    
    \begin{figure}[h!]
        \centering
        \includegraphics[width=\textwidth]{Images/03_methods/blip2_cosine_sim.png}
        \caption{System architecture}
        \label{fig:system_overview}
    \end{figure}

    \item Dynamic update of value maps as new observations are integrated.
\end{itemize}

\section{Persistent Semantic 3D Mapping} \label{sec:persistent_mapping}

\textbf{Global Map Construction with Open-Fusion}
\begin{itemize}
    \item Incremental creation of a global semantic point cloud map integrating RGB-D observations over time.
    \item Registration of observations using robot poses to maintain a consistent world representation.
    \item Association of semantic labels with 3D points based on query relevance scores.
\end{itemize}

\textbf{Semantic Clustering and Graph Node Generation}
\begin{itemize}
    \item Clustering of points with similar semantic labels to form object-level hypotheses.
    \item Construction of semantic graph nodes representing detected object instances with aggregated confidence scores.
    \item Maintenance of the semantic graph as a persistent memory for multi-object search tasks.
\end{itemize}

\section{Promptable Zero-Shot Detection} \label{sec:zero_shot_detection}
\begin{itemize}
    \item In this work YOLO-E \cite{yolo_e} is used as the promptable zero-shot detection model.
    \item YOLO-E has the following advantages:
    \begin{itemize}
        \item High inference speed suitable for real-time applications.
        \item Ability to handle open-vocabulary object detection based on text prompts.
        \item Integration of both visual and textual information for robust detection.
        \item Pre-trained on large-scale datasets, enabling zero-shot generalization to unseen object categories.
    \end{itemize}
\end{itemize}


\textbf{Open-Vocabulary Object Detection with YOLO-E}
\begin{itemize}
    \item Utilization of the YOLO-E model for open-vocabulary object detection based on text prompts.
    \item Extraction of 2D bounding boxes and associated confidence scores for detected objects.
    \item Segmentation of detected objects to isolate relevant pixels for 3D localization.
\end{itemize}

\textbf{Depth-Based 3D Localization}
\begin{itemize}
    \item With camera intrinsics and depth information, the 2D bounding boxes are projected into 3D space.
    \item Calculation of 3D coordinates for each detected object using depth values within the bounding box.
    \item Semantic detection pointclouds are passed are then clustered and the centroid of each cluster is computed to obtain robust 3D object locations.
    \item For each cluster, the mean of the confidence scores of the associated 2D detections is calculated to assign a confidence score to the 3D localization.
\end{itemize}

\begin{figure}[h!]
    \centering
    \includegraphics[width=1.0\textwidth]{Images/03_methods/vlm_detector.png}
    \caption{YOLO-E detection to graph node 3D localization}
    \label{fig:vlm_detector}
\end{figure}


\section{Fusion Strategy}\label{sec:fusion_strategy}
\textbf{Exploration–Memory Weighting}
\begin{itemize}
    \item Exploration and memory graph nodes are fused and weighted as follows:
    \begin{itemize}
        \item Proximity weighting: Nodes closer to the robot's current position are given higher weights.
        \item Exploration vs Memory: Nodes from the exploration source are prioritized over memory nodes to encourage discovery of new information.
        \item Costmap weighting: Nodes located in areas with lower navigation costs are favored to optimize path planning and navigation efficiency.
    \end{itemize}
\end{itemize}

\textbf{Multi-Source Detection Fusion}
\begin{itemize}
    \item Detection graph nodes are weighted based on:
    \begin{itemize}
        \item YOLO-E confidence scores: Higher confidence detections are given more weight.
        \item BLIP-2 value map: Detections with higher semantic relevance to the text prompt are prioritized.
        \item The nearer detection graph nodes are to memory graph nodes, the higher their weight.
    \end{itemize}
\end{itemize}

\textbf{Relevance Filtering and Node Suppression}
\begin{itemize}
    \item Each source's graph nodes are filtered based on a relevance threshold to eliminate graph nodes within the fov map.
    \item Relevance map is build over time
    \item If a graph node is located in an area that has already been explored and found to be irrelevant to the prompt, it is suppressed.
\end{itemize}


\section{Behavior Tree for Semantic-Guided Exploration} \label{sec:behavior_tree}

\textbf{High-Level Task Structure}
\begin{itemize}
    \item The behavior tree (BT) is designed to manage the high-level task structure for semantic-guided exploration.
    \item The BT consists of the following main components:
    \begin{itemize}
        \item Initialization: Clearing Maps, Publishing Prompts
        \item Detection Branch: If object is detected over a threshold, navigate to it, realign to object take picture
        \item Exploration Branch: While object not detected, perform semantic frontier exploration navigating to highest valued frontiers or memory nodes
        \item Termination: If object found, end mission; If time limit reached, end mission
        \item Behavior tree is called with a ros2 action server, which returns on termination success or failure, and actual path taken
    \end{itemize}
\end{itemize}

\begin{figure}[h!]
    \centering
    \includegraphics[width=0.5\textwidth]{Images/03_methods/sage_bt_flowchart.png}
    \caption{System architecture}
    \label{fig:system_overview}
\end{figure}

\textbf{Integration with Navigation Stack}
\begin{itemize}
    \item Navigation stack used for low-level path planning and obstacle avoidance.
    \item Action used: navigate\_to\_pose, Spin
\end{itemize}
    
